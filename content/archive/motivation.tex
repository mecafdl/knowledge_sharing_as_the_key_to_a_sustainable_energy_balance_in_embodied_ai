\section{Introduction}
% SUBSECTION ========================================================================================
\subsection{Embodied AI and the smart factory}
%---
\begin{figure*}[!h]
	\centering
	\includegraphics[width=0.90\textwidth]{fig/smart_fabrik.pdf}
	\caption{The smart factory.}
	\label{fig:smartFactory}
\end{figure*}
%---
%---
\begin{fancyquotes}
	AI is one of the cornerstones of the growing digitization of industry ('Industry 4.0'). Technologies underpinning  this  process  ---  such as IoT,  5G,  \textbf{cloud  computing},  big  data  analytics,  \textbf{smart  sensors},  augmented  reality,  3D  printing  and  \textbf{robotics}  ---  are  likely  to  transform  manufacturing  into  a  single cyber-physical  system  in which digital  technology,  internet  and  production  are merged in one \cite{szczepanski_2019}.
\end{fancyquotes}
%---
Classical AI interprets intelligence as a purely computational symbol processing problem decoupled from corporeal agents. Some of the recent and most outstanding results from this perspective are exemplified in the progress in classification, natural language processing, image and pattern recognition that machine learning, one of its sub disciplines, has produced. In this paper, however, we focus on an expanded view; \emph{embodied AI}. This view stems out from the perspective that intelligence necessitates from a body to develop and that it results from the constant interaction of a physical agent with its environment, via sensing and acting. Therefore, from an engineering point of view, embodied AI requires physically embodied systems with sensor and actuation capabilities \cite{ziemke2004embodied}, with robots being one such example and the focus of attention in this paper.

In the near future, smart factories will come about as Industry 4.0 continues to develop. As a result, the norm will be networks of factories with ubiquitous computing capabilities that share information collected by their distributed sensors spread across systems and machines. Flexibility will be an defining characteristic of these factories and will make it possible to adapt to changing production demands and novel manufacturing methods. Modern industrial and collaborative robots (\emph{cobots}), equipped with a myriad of sensors and local computing capabilities, will be a core elements to enable flexibility in the smart factories. As a consequence the numbers of robot units employed will multiply and take care of diverse manufacturing tasks either on their own or in collaboration with humans. In turn, this will motivate the further development and implementation of cloud robotics to run running state-of-the-art embodied AI algorithms, exploiting the large computational, storage and communications capabilities of modern data centers, to enable the exchange and leverage of experience collected by the robots as they interact with their environment and with the goal of constantly improving the way tasks are executed, see Fig~\ref{fig:smartFactory}.

% SUBSECTION ========================================================================================
\subsection{The problem with AI}
Artificial Intelligence has applications beyond the realm of the smart factory with potential to face global climate and environmental problems\cite{pwcreport}. Improvement in decision-making for environmental protection, reduction of energy and resource demand, or robots that clear plastic from rivers and oceans are but a few examples of the numerous areas of application. Nonetheless, achievements in classical and embodied AI are rooted in complex training mechanisms, which in turn require computing power and server space which in turn mean elevated energy consumption. The impact of this increment can be tracked down to an increased consumption of environmental and financial resources. The United Nations environment program addresses such issues with the term \emph{maladaptation}: ``an adaptation that does not succeed in reducing vulnerability but increases it instead''\cite{un2019emerging}. As an example, the potential financial and environmental costs of training artificial intelligence used to monitor and improve climate change and environmental impact are most likely higher, than the expected benefits\cite{Strubell2019EnergyAP}. Such can be the case of embodied AI in the smart factory when on top of the computational demand the energetic cost of the complete system are taken into account (e.g. originated by the varied and numerous types of robots used for manufacturing). 

As AI and AI applications are considered to be the most promising technologies to improve work, health, mobility and the well-being of society in general in the near future, three major issues need to be addressed:
\begin{enumerate}
	\item Energy consumption: given the increased abilities of AI algorithms the electricity demand of required for their training will massively increase, potentially even double or triple. The situation is aggravated by the fact that embodied AI (i.e. AI that is coupled together with a physical system ---a robot in our case---) needs additional power for motion to function properly
	\item Growth: use of these technologies will increase significantly as AI and AI applications become part of everyday life
	\item Recyclability: technologies are constantly evolving and building new robots and systems consumes a substantial amount of resources. We have to make sure that they are recyclable in order to avoid waste and conserve spare resources
\end{enumerate}
 
To ensure not only energy efficient but resource aware AI and AI applications a new holistic methodological approach is needed. \hl{Showing the carbon footprint of digitization, the production and delivery of digital content as well as the devices on which they are accessed}. \textcolor{red}{This article addresses the issue of evaluating these costs by introducing key performance indicators illustrating energy efficiency of AI and AI applications which are considered to be the next step to efficient machine learning. Stating that energy efficient algorithms are just the beginning, to avoid maladaptation the complete system has to be properly evaluated.}
\hl{The main purpose of this publication is to raise awareness regarding the necessary change of paradigm for the community}.