\section*{Appendices}
\begin{appendices}
	
% ===================================================================================================
%                                                 |                                                 |
%                                                 |                                                 |
% -------------------------------------------- SECTION ---------------------------------------------|
%                                                 |                                                 |
%                                                 |                                                 |
% ===================================================================================================
\section{Industrial robot energy consumption support data}\label{sec:app_robot_ener_consumption}
According to \cite{montaqim2015} and available press releases of different robotic companies \cite{fanuc2015, yaskawa2014, ABB2015}, the approximate distribution of the industrial robot install base per manufacturer is shown in Fig.~\ref{fig:manufacturers_pie}.
% ---
\begin{figure}[!ht]
	\centering
	\includegraphics[width=0.95\columnwidth]{fig/manufacturers}
	\caption{Percentage of installed industrial robots per manufacturer}
	\label{fig:manufacturers_pie}
\end{figure}
% ---
Since Fanuc, Yaskawa, and ABB make for two-thirds of the total install base of industrial robots, we took the power consumption of the robots from those manufacturers to estimate the total power consumption. 
After surveying the datasheets for their robot portfolio, the average power consumption for each model was estimated. Additionally, every manufacturer classifies their robots according to one or more possible applications, which can be grouped into the application categories defined by the IFR. The average power consumption was calculated for every application using the values reported in the robot datasheets. Finally, the power consumption for each category was computed as a weighted average based on the companies' market share percentage (assuming that 68 \% is the total number of robots)\footnote[1]{These numbers should be used with discretion since there is no available information on which are the most common installed robot models. This information may change the estimation.}. The estimated power consumption per robot application is shown in Fig.~\ref{fig:ir_average_power}.
%---
\begin{figure}[h]
	\centering
	\includegraphics[width=0.95\columnwidth]{fig/industrial_robots_average_power_per_category}
	\caption{Average power consumption of industrial robots per category.}
	\label{fig:ir_average_power}
\end{figure}
%---
Using these numbers and the estimated operational stock of industrial robots reported in \cite{statista_ir_operational_stock} and by the International Federation of Robotics (see Fig.~\ref{fig:ir_stock}), the estimated worldwide industrial robot energy consumption was computed and shown in Fig.~\ref{fig:ir_energy}.
% ===================================================================================================
%                                                 |                                                 |
%                                                 |                                                 |
% -------------------------------------------- SECTION ---------------------------------------------|
%                                                 |                                                 |
%                                                 |                                                 |
% ===================================================================================================
\section{Collaborative robots}\label{sec:app_cobot_ener_consumption}
To approximate the energy consumption of cobots we looked at the power consumption per payload of various manufacturers, see Fig.~\ref{fig:cobot_watt_per_kg} resulting in an average power consumption of approximately 40 W. Together with a typical power consumption of the robot controller of 60 W \cite{Heredia2023BreakingEnergyConsumption}, we consider a total of 100 W power demand. Similar to the industrial robots, the worldwide energy consumption was calculated assuming a 24/7 operation.
%---
\begin{figure}[h]
	\centering
	\includegraphics[width=0.95\columnwidth]{cobot_watt_per_kg.png}
	\caption{Power consumption per payload for different cobots.}
	\label{fig:cobot_watt_per_kg}
\end{figure}
%---
Information taken from \url{https://www.statista.com/statistics/748128/estimated-collaborative-robot-sales-worldwide/}
\begin{figure}[!ht]
	\centering
	\includegraphics[width=0.4\textwidth]{fig/cobots_global_sales_2018_2025}
	\caption{Projected sales of collaborative robots worldwide from 2018 to 2025 (in 1,000 units)}
	\label{fig:cobots_sales_projection}
\end{figure}

Taken from \url{https://www.statista.com/statistics/1044767/collaborative-robots-market-by-application/}
\begin{figure}[!ht]
	\centering
	\includegraphics[width=0.4\textwidth]{fig/global_market_size_cobots_by_application}
	\caption{Global market for collaborative robots in 2017, by application.}
	\label{fig:global_market_size_cobots_by_application}
\end{figure}

\section{Service robots}
\begin{figure}[h]
	\centering
	\includegraphics[width=0.4\textwidth]{fig/service_robots_professional_use_main_app_sales}
	\caption{Unit sales of service robots for professional use.}
	\label{fig:service_robots_professional_use_main_app_sales}
\end{figure}

\begin{figure}[h]
	\centering
	\includegraphics[width=0.4\textwidth]{fig/service_robots_professional_main_app_value}
	\caption{Estimated market value of service robots for professional use according to their main application.}
	\label{fig:service_robots_professional_main_app_value}
\end{figure}

\begin{figure}[h]
	\centering
	\includegraphics[width=0.4\textwidth]{fig/service_robots_professional_use_main_other_sales}
	\caption{Unit sales for service robots for other professional use.}
	\label{fig:service_robots_professional_use_main_other_sales}
\end{figure}
	
	% ---	
	\begin{figure}[!t]
		\centering
		\includegraphics[width= 0.9\columnwidth]{fig/advanced_robots_in_manufacturing_projected_global_demand}
		\caption{Projected demand for advanced robotics in manufacturing worldwide between 2018 and 2021 (in billion USD).}
		\label{fig:advanced_robots_in_manufacturing_projected_global_demand}
	\end{figure}	
	% ---
\end{appendices}

% Notes:
% Up to now: Time is saved by increasing resources (more computation power, means faster)
% This ultimately leads to disaster (the energy is simply not there)
% It is not enough to reduce the individual energy consumption
% Solution to systematically and independent of the particular technology to reduce both energy and time at the same time is collective learning based on transfer learning

% Task energy cannot be reduced. The body needs that energy times efficiency. Even if efficiency is 1 -> too many tasks -> too much energy
% Thought experiment
% Learning needs a lot of energy (executing and computing)
% Relevance

% Deprecated introduction: