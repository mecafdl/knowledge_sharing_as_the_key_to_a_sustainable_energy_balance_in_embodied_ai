% ===================================================================================================
%                                                 |                                                 |
%                                                 |                                                 |
% -------------------------------------------- SECTION ---------------------------------------------|
%                                                 |                                                 |
%                                                 |                                                 |
% ===================================================================================================
\section{Energy demand of Embodied AI}\label{sec:energy_demand_embodied_ai}
The benefits embodied AI will bring about as it permeates many areas, such as the smart factory, healthcare services, and people's homes, will take a toll, particularly on the energy demand. We briefly go over some recent works that have looked at this issue and that focus on either of the two most relevant components of embodied AI: machine learning and robotics, with the former directly connected to the CCE energy usage category and the latter corresponding to the BBE and MIE categories.

% SUBSECTION ========================================================================================
\subsection{Energy demand in machine learning}
%\myhl{Artificial Intelligence (AI) and machine learning (ML) have revolutionized numerous industries, enabling unprecedented advancements in technology, healthcare, finance, and more. However, this exponential growth in AI applications comes at a cost: an increasingly substantial demand for energy. AI and ML require vast amounts of computational power to process, analyze, and learn from data. As AI applications grow in complexity, so does the computational workload, leading to a surge in energy consumption. Massive data centers and powerful hardware, such as Graphics Processing Units (GPUs), are essential components of AI infrastructure but contribute significantly to energy consumption.
%}


\myhl{The unprecedented advancements in many environments unleashed by the exponential growth in AI and machine learning applications have come at the expense of the vast amounts of computational power to process, analyze, and learn from data. The consequential rise in  energy consumption is associated to the heavy computational workloads of data centers and related hardware, such as Graphics Processing Units (GPUs). Recently, there has been an increasing concern in the research community about the negative impact of artificial intelligence and machine learning on the environment. For instance, works such as \cite{schwartz2019green}, \cite{vinuesa2020role}, and \cite{Strubell2019EnergyAP} discuss the efficiency of computation-intensive deep learning algorithms\footnote{Consider that the number of computations required by deep learning algorithms has increased more than 300000x in the last decade \cite{schwartz2019green}.} --- such as natural language processing (NLP) --- and express their impact on the environment based on the carbon footprint left by the data centers used to train and run them. Related works have defined metrics to assess the energy consumption of machine learning algorithms, such as the energy efficiency of their development phases \cite{zhou2020hulk},  the accuracy, model size, time, and CPU/GPU energy consumption for the training and inference phases \cite{Dalgren2019GreenMLA}, as well as other system level performance counters like real-time, instruction-level, and a hardware-level power estimation \cite{garcia2019estimation}. Despite this rising awareness about energy consumption in AI, actions are yet to be taken to understand the roots of the problem and present potential solutions to alleviate it.
}



% SUBSECTION ========================================================================================
\subsection{Energy demand in robotics}\label{sec:energy_in_robotics}
Recent statistics \cite{IFR2019} shows a clear upward trend in the worldwide demand for industrial robots. Such a trend directly links to a significant increase in electric energy consumption. For instance, although focused only on the robotics market in the United States, the study in \cite{barnett_2017} projects that 22,822 GWh will be demanded by the ever-increasing introduction of robots in many aspects of human life. Yet, although the reduction of the energy consumption from industrial robots has been studied in works like \cite{schroder2014, chalmers2015, mohammed2014, chemnitz2011} providing alternatives to make robots more energy efficient ---e.g., exploiting robots with elastic actuation \cite{scalera2019natural, carabin2017review, bolivar2017general, haddadin2011optimal,haddadin2012intrinsically} or methods for better hardware selection and storage and sharing of energy and optimized motion planning \cite{carabin2017review}---,  there is still no precise estimate of the energy demand rise that the increasing number of robots in operation will bring into the picture nor a strategy to address this problem.

Identifying the necessity to provide a holistic understanding of the energy challenges pertaining the advent of these many potential embodied AI systems\footnote{A report by Next Move Strategy Consulting shows an exponential increase of the worldwide market size of AI-driven robots \cite{statista_ai_robots_market_size}.}, we attempt to contribute to this understanding by elaborating on these challenges and positioning the recent paradigm of collective learning as the core strategy that can leverage scalability and knowledge exchange to achieve energy efficiency in embodied AI.